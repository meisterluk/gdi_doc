\chapter{Mengenlehre}
%
Die Mengenlehre ist ein Teilgebiet der Mathematik und beschäftigt sich mit Sammlungen von unterscheidbaren Objekten. Da in der Informatik das Konzept der Mengen oft wiederverwendet wird, sei eine oberflächliche, aber präzise Betrachtungsweise gegeben.

Eine \emph{Menge} (engl. Set) ist
\begin{itemize}
 \item ungeordnet
 \item jedes Element unterscheidet sich von anderen Elementen in dieser Menge und kommt daher nur einmal vor
 \item besitzt eine beliebige Anzahl von Elementen
\end{itemize}
%
Die leere Menge wird mit $\diameter$ oder $\{\}$ bezeichnet. Die Anzahl der Elemente einer Menge nennt man auch \emph{Kardinalität einer Menge}. Die Definition erlaubt sowohl endliche als auch unendliche Mengen. Mengen werden wie Matrix meist mit einem Großbuchstaben bezeichnet.
%
\[
   A = \{1, 2, 3, 4\}
\] \[
   B = \{\}
\]

Durch einen vertikalen Balken können weitere Kriterien angeführt werden:
%
\[
  C = \{x^2 | x \in \mathbb{N}
\]

\section{Eindeutigkeit und Abgrenzungen zu Synonymen}

Entsprechend der genannten Definition darf jedes Element genau nur einmal vorkommen. Obwohl diese Definition oft respektiert wird, wird sie im Allgemeinen inkonsistent verwendet. So wird auch die Sequenz ($\{1, 2, 3, 2\}$) als Menge bezeichnet. Die Frage ist, wie sich die Begriffe Liste, Sequenz, Tupel und Menge voneinander abgrenzen.

\begin{description}
 \item[Liste] Eine Liste ist eine geordnete Sammlung von beliebig vielen Elementen.
 \item[Sequenz] Eine Sequenz ist eine geordnete Sammlung von beliebig vielen Elementen,
                die meisten auf eine endliche Größe beschränkt sind.
 \item[Tupel] Eine geordnete Sammlung von homogenen Elementen endlicher Anzahl.
 \item[Menge] Ungeordnete Sammlung von unterschiedlichen Objekten.
\end{description}

Alle Begriffe erlauben grundsätzlich Verschachtelungen (zB Liste in Liste),
werden jedoch meist nur eindimensional verwendet.

%TODO: \section{Zusammenhang der Logik mit der Mengenlehre}
%
%Vereinigungsmenge. Durchschnitt.
%
%\section{Mengen Notation}
%%
%Im Umgang mit Mengen werden mehrere Symbole verwendet, die das Verhältnis zwischen Mengen beschreiben.
%Es seien drei Mengen $A = \set{1, 3, 4}$, $B = \set{2, 4, 6}$ und $CC = {NPG
%%
%\[
%   A \subset B
%   A \subseteq B
%   A \superset B
%   A \superseteq B
%\]
