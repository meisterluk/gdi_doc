\chapter{Bekannte Informatiker}
\index{Informatiker}
%
Diese Liste erhebt keinerlei Anspruch auf Vollständigkeit und hebt nur Teile der Arbeiten jeweiliger Personen hervor.
%
\begin{description}
  \item[John Backus \life{1924}{2007}] \hfill{} \\
    Leitung der Entwicklung der Programmiersprache Fortran.
    Bekannt für seine Backus-Naur-Form im Bereich der formalen Grammatiken.
    Forschte anschließend im Bereich der funktionalen Programmierung.

   \item[Tim Berners-Lee \born{1955}] \hfill{} \\
    Entwickelte am CERN den ersten Webserver und begründete damit das WWW.
    Bemüht sich um die Standardisierung von HTTP, HTML und anderen Technologien mithilfe seines W3C.
    Heute Fokus auf XML-Standards und seine Vision des ,,Semantic Web``.

  \item[George Boole \life{1815}{1864}] \hfill{} \\
    Untersuchte und entwickelte jenen Logikkalkül, der Grundlage der modernen Logik heute ist.
    Ihm zu Ehren sprechen wir von boolschen Variablen und boolschen Formeln.

  \item[Noam Chomsky \born{1928}] \hfill{} \\
    Linguist, welcher durch Untersuchung der formalen Grammatiken eine Klassifikation (,,Chomsky-Hierarchie``) der Sprachen nach ihrer Mächtigkeit erreichte.
    Trug damit erheblich zur Weiterentwicklung des Compilerbaus bei.
    Politisch aktiv.

  \item[Edsger Dijkstra \life{1930}{2002}] \hfill{} \\
    Wesentliche Beiträge im Bereiche der Algorithmen (Graphentheorie, Ressourcenverwaltung).
    Prägte den Begriff ,,strukturierte Programmierung`` und half Programmierung zugänglich zu machen.
    Seine Notizen (,,EWDs``) sind heute noch wichtigstes Referenzmaterial.

  \item[Grace Hopper \life{1906}{1992}] \hfill{} \\
    Als eine der Pioniere der Informatik entwickelte sie den ersten Compiler, verbreitete den Term ,,Debugging`` und half COBOL zu entwickeln.

  \item[Alan Kay \born{1940}] \hfill{} \\
    Alan Kay begeistert sich für graphische Oberflächen und sah den entsprechenden Bedarf an objektorientierten Ansätzen zur Steuerung dieser.

  \item[Donald Knuth \born{1938}] \hfill{}
    Autor des Textsatzprogramms \TeX. Bekannt für seine Beiträge im Bereich der Algorithmen (zB Knuth–Morris–Pratt Algorithmus). Arbeitet perfektionistisch seit 1962 an seiner Bücherserie ,,The Art of Computer Programming``.

  \item[Gottfried Leibniz \life{1646}{1716}] \hfill{} \\
    Aufgrund seiner Betrachtung des binären Zahlensystems wird er als Mathematiker mit der Informatik in Verbindung gebracht.

  \item[J.C.R. Licklider \life{1915}{1990}] \hfill{} \\
    Wesentliche Beiträge in den frühen Zeiten der künstlichen Intelligenz.
    Weitere Forschung auch in den Bereichen Mensch-Maschine-Interaktion.

  \item[John McCarthy \life{1927}{2011}] \hfill{} \\
    Durch die Schaffung der Sprachfamilie LISP ermutigte er zahlreiche Entwickler für die Artificial Intelligence, die er wesentlich mitgeprägt hatte und einen Hype in den 60/70er Jahren auslöste.

  \item[Ted Nelson \born{1937}] \hfill{} \\
    Ted Nelson trägt als Technikphilosoph zur Informationstechnologie bei und begleitete etwa die Entwicklung des Hypertexts von Tim Berners-Lee mit und beobachtet Entwicklungen im Internet.

  \item[Dennis Ritchie \life{1941}{2011}] \hfill{} \\
   Gemeinsam mit Ken Thompson entwickelte er das UNIX-System, welches Grundlage für zahlreiche Betriebsysteme bildet und einen gemeinsamen Standard schaffte.

  \item[Claude Shannon \life{1916}{2001}] \hfill{} \\
   Claude Shannon gilt mit seiner Formulierung des Begriffs der Entropie als Begründer der Informationstheorie und trug mit Grundlagenforschung zur Übertragung von Signalen wesentlich zur Informationstechnologie bei.

  \item[Richard Stallman \born{1953}] \hfill{} \\
    Mit der Schaffung des Gnu-Projekts gilt er als Hauptfigur der Freie-Software-Bewegung.

  \item[Ivan Sutherland \born{1938}] \hfill{} \\
    Vorreiter und aktiver Wissenschafter im Bereich der Computergraphik (zB Cohen-Sutherland-Algorithmus).

  \item[Andrew Tannenbaum \born{1944}] \hfill{} \\
    Andrew Tanenbaum ist bekannt geworden durch Beiträge im Bereich der Betriebsysteme und Netzwerke.

  \item[Ken Thompson \born{1943}] \hfill{} \\
   Gemeinsam mit Dennis Ritchie entwickelte er das UNIX-System, welches Grundlage für zahlreiche Betriebsysteme bildet und einen gemeinsamen Standard schaffte.

  \item[Alan Turing \life{1912}{1954}] \hfill{} \\
   Als Begründer der theoretischen Informatik formalisierte er die Berechenbarkeit für die Informatik. Mit der Turingmaschine, dem Turingtest und der Vision Artificial Intelligence legte er den Grundstein für wichtige Automatisierungsprobleme.

  \item[Konrad Zuse \life{1910}{1995}] \hfill{} \\
   Konrad Zuse entwickelte die Z3, welche den ersten frei programmierbaren Computer darstellt und legt damit einen Grundstein für die Architektur von Rechenmaschinen.

  \item[Niklaus Wirth \born{1934}] \hfill{} \\
   Bekannt als Mitentwickler wichtiger prozeduraler Programmiersprachen. Neben Euler, Modula-2 und Algol W ist er berühmt für das Schaffen der Programmiersprache Pascal mit welchem Generationen von Programmierern ausgebildet wurden.

  \item[Alan Kay \born{1940}] \hfill{} \\
   Mit seinen Arbeiten an graphischen Oberflächen und den daraus entstandenem objektorientierten Programmierparadigma trug er zur Entwicklung des Personal Computer bei.

  \item[Edgar F. Codd \life{1923}{2003}] \hfill{} \\
   Er gilt als der Erfinder relationaler Datenbanken und prägte deren Theorie wesentlich mit. Da relationale Datenbanken noch heute die meist genutzten sind, findet sich Schaffen heute noch Anwendung.
\end{description}
