\begin{titlepage}
  \begin{center}
    \begin{figure}[ht!]
      \begin{center}
        \includegraphics[width=88px,height=33px]{img/cc0.pdf} \\[20pt]%
        %
        To the extent possible under law, Lukas Prokop has waived all copyright and related
        or neighboring rights to ``GDI Skriptum''. \\
        This work is published from: Österreich. 
      \end{center}
    \end{figure}
    %
    \vspace{50pt}
    %
    \noindent Fehler bitte an
    \href{mailto:admin@lukas-prokop.at}{\nolinkurl{admin@lukas-prokop.at}}
    melden. Danke!
    %
    \vspace{50pt}

    Dieses Dokument wurde für die Lehrveranstaltung \courselfocs{}
    (Technische Universität Graz, Prof. Wolfgang Slany) konzipiert
    und deckt die wesentlichen Inhalte ab. Es soll einerseits als
    Leitfaden zu den Inhalten und Themen dienen und andererseits an Beispielen
    Konzepte und Werkzeuge der Informatik verständlich machen. Es sei zu beachten,
    dass die Inhalte keineswegs die Themen so vollständig und tief erfassen
    wie sie für Wirtschaft und Forschung notwendig sind. Es sei ein Fokus
    auf die \emph{Grundlagen der Informatik} gegeben.

    \vspace{50pt}
    Ich danke folgenden Personen, die durch Feedback zur Verbesserung
    des Dokuments beigetragen haben: \\[15pt]
    criscom, Frédéric Gierlinger, Georg Regitnig, Wolfgang Slany, Karl Voit,
    Helmut Zöhrer
  \end{center}
\end{titlepage}

