\chapter{Graphentheorie}
%
Ein Graph ist eine Menge $(V, E)$ von Knoten (engl. vertex) und Kanten (engl. edge). Graphen unterscheiden sich aufgrund ihrer Eigenschaften und damit assoziiert werden zahlreiche neue Begriffe definiert, um den Austausch innerhalb des Forschungsgebiets zu erleichtern.
%
\begin{description}
  \item[(un)gerichteter Graph]
    Ein gerichteter Graph unterscheidet zwischen einer Kante $(u, v)$ und der Kante $(v, u)$, während diese Unterscheidung beim ungerichteten Graphen wegfällt.
  \item[Zyklus]
   Eine beliebige Sequenz von Knoten, die einem Ablauf von Kanten folgen, wobei der Startknoten gleich dem Endknoten ist.
  \item[Schleife]
   Genau eine Kante, die den Knoten mit sich selbst verbindet. Sie ist auch damit ein Zyklus, wobei die Sequenz eine Maximallänge von 1 anlegt.
  \item[Baum]
   Bäume sind das Lieblingsspielzeug der Informatik. Ein Baum wird als zusammenhängender, zyklusfreier Graph verstanden. Er ermöglicht es Daten in einer hierarchischen Baumstruktur abzubilden.
  \item[Brücke]
   Unter einer Brücke versteht man eine Kante, die bei Entfernung die Eigenschaft ,,zusammenhängend`` des Graphen zu ,,unzusammenhängend``.
  \item[Clique]
   Unter Clique versteht man eine Menge von Knoten $k$, in der jeder Knoten direkt mit jedem anderen verbunden ist. Wird die Clique für den gesamtem Graphen angewandt, fällt die Definition mit der Definition des vollständigen Graphens zusammen.
  \item[Eulerscher Graph]
   Der Eulersche Graph ist ein spezieller Zyklus, der alle Knoten des Graphen nutzt und jede Kanten genau nur 1mal verwendet.
  \item[Grad]
   Der Grad eines Knotens ist definiert als die Anzahl an ein- und ausgehenden Kanten.
  \item[Hamiltonpfad]
   Unter einem Hamiltonpfad versteht man einen Pfad im Graphen, der jeden Knoten genau einmal besucht.
  \item[Pfad]
   Unter einem Pfad versteht man eine Sequenz von Kanten, die einer Wanderung durch den Graphen entspricht (Endknoten sind Startknoten des nächsten Knotens)
  \item[Planarität]
   Ein Graph ist planar, wenn er ohne Überschneidungen von Kanten auf einer 2D-Fläche gezeichnet werden kann.
  \item[Quelle/Senke]
   Unter der Quelle versteht man bei einem gerichteten Graph einen Knoten, der nach außen führende Kanten besitzt. Eine Senke ist ein Knoten, der nur eingehende Kanten auf diesen Knoten besitzt.
  \item[Vollständiger Graph]
   Unter einem vollständigen Graph versteht man einen Graph, in dem jeder Knoten mit jedem Knoten direkt verbunden ist.
\end{description}

%TODO: \section{Konzept Adjazenzmatrix}
