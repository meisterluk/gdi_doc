\chapter{Abstraktion}
%
Die Informatik kombiniert all die genannten Grundlagen um höhere Konzepte
zu bauen. Dieser Ansatz nennt sich Abstraktion. Durch das Auslassen kleiner
Details auf der unteren Abstraktionsebene, können wir ein vereinfachtes Bild
in der aktuellen Abstraktion annehmen.
Ein Beispiel ist etwa das Schienennetz Österreichs. Möchte ein Informatiker
alle freigegebenen und besetzen Blöcke visualisieren, so interessiert ihn
das einzelne Material des Gleises auf einem gewissen Streckenabschnitt
wenig. Viel wichtiger ist, dass das Netz als ein Graphen abgebildet werden
kann. Die Visualisierung ist eine abstrakte Sichtweise auf das Netz.

\section{In der Programmierung}
%
Auch in der Programmierung benötigen wir durchgehend Methoden zur Abstraktion.
Dabei werden einzelne Instruktionssequenzen zu einem Block zusammengefasst,
welcher eine gewisse semantische Funktionalität repräsentiert. So kann etwa
eine kleine Unterfunktion eine Konvertierung der ASCII-Kleinbuchstaben in
Großbuchstaben vornehmen.

\begin{verbatim}
int uppercase(char *string)
{
    int index;
    for (index=0; index<strlen(string); index++) {
        if (string[index] >= 97 && string[index] <= 122)
            string[index] = string[index] & (~32);
    }
    return 0;
}
\end{verbatim}


